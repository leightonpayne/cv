%%%%%%%%%%%%%%%%%%%%%%%%%%%%%%%%%%%%%%%%%%%%%%%%%%%%%%%%%%%%%%%%%%%%%%%%%%%%%%%
% A clean template for an academic CV
%
% Uses tabularx to create two column entries (date and job/edu/citation).
% Defines commands to make adding entries simpler.
%
%%%%%%%%%%%%%%%%%%%%%%%%%%%%%%%%%%%%%%%%%%%%%%%%%%%%%%%%%%%%%%%%%%%%%%%%%%%%%%%

\documentclass[10pt, a4paper]{article} % Document format
\usepackage[utf8]{inputenc} % Unicode support

% Personal information
\newcommand{\Title}{Curriculum Vit\ae}
\newcommand{\FirstName}{Leighton}
\newcommand{\LastName}{Payne}
\newcommand{\MyName}{\FirstName\ \LastName}
\newcommand{\Initials}{LJ}
\newcommand{\Me}{\textbf{\LastName, \Initials}}
\newcommand{\Email}{leighton.payne@postgrad.otago.ac.nz}
\newcommand{\PersonalWebsite}{leightonpayne.com}
\newcommand{\ORCID}{0000-0003-2305-6827}

% Affiliations
\newcommand{\UOO}{University of Otago}

% Current affiliation
\newcommand{\Company}{\UOO}
\newcommand{\Department}{Microbiology and Immunology}
\newcommand{\Address}{720 Cumberland Street \\ Dunedin 9045, New Zealand}
\newcommand{\Affiliation}{\Department \\ \Company \\ \Address}

% Coauthors
\newcommand{\SAJ}{Jackson, SA}
\newcommand{\TCT}{Todeschini, TC}
\newcommand{\YW}{Wu, Y}
\newcommand{\BJP}{Perry, BJ}
\newcommand{\CWR}{Ronson, CW}
\newcommand{\PCF}{Fineran, PC}
\newcommand{\FLN}{Nobrega, FL}
\newcommand{\SM}{Meaden, S}
\newcommand{\MRM}{Mestre, MR}
\newcommand{\CP}{Palmer, C}
\newcommand{\NT}{Toro, N}
\newcommand{\THughes}{Hughes, T}

% Template configuration
%%%%%%%%%%%%%%%%%%%%%%%%%%%%%%%%%%%%%%%%%%%%%%%%%%%%%%%%%%%%%%%%%%%%%%%%%%%%%%%

\usepackage{needspace}

% Disable hyphenation
\usepackage[none]{hyphenat}

% Control the font size
\usepackage{anyfontsize}

% Icon fonts (requires using xelatex or luatex)
\usepackage[fixed]{fontawesome5}
\usepackage{academicons}

% Template variables for styling
\newcommand{\TablePad}{\vspace{-0.4cm}}
\newcommand{\SoftwareTitle}[1]{{\bfseries #1}}
\newcommand{\TableTitle}[1]{{\fontsize{12pt}{0}\selectfont \itshape #1}}

% For fancy and multipage tables
\usepackage{tabularx}
\usepackage{ltablex}

% Define a new environment to place all CV entries in a 2-column table.
% Left column are the dates, right column the entries.
\usepackage{environ}
\NewEnviron{EntriesTable}{
\TablePad
\begin{tabularx}{\textwidth}{@{}p{0.10\textwidth}@{\hspace{0.02\textwidth}}p{0.88\textwidth}@{}}
  \BODY
\end{tabularx}
}
\NewEnviron{EntriesTableExtra}{
\TablePad
\begin{tabularx}{\textwidth}{@{}p{0.10\textwidth}@{\hspace{0.02\textwidth}}p{0.79\textwidth}@{\hspace{0.02\textwidth}}>{\raggedright\arraybackslash}p{0.07\textwidth}}
  \BODY
\end{tabularx}
}

% Macros to add links and mark publications
\newcommand{\DOI}[1]{doi:\href{https://doi.org/#1}{#1}}
\newcommand{\DOILink}[1]{\href{https://doi.org/#1}{doi.org/#1}}
\newcommand{\Website}[1]{\href{https://#1}{#1}}
\newcommand{\Print}[1]{\href{https://doi.org/#1}{\faFilePdf}}
\newcommand{\GitHub}[1]{\href{https://github.com/#1}{\faGithub}}
\newcommand{\Data}[1]{\href{https://doi.org/#1}{\faChartLine}}
\newcommand{\Slides}[1]{\href{https://#1}{\faTv}}
\newcommand{\SlidesDOI}[1]{\href{https://doi.org/#1}{\faTv}}
\newcommand{\PosterDOI}[1]{\href{https://doi.org/#1}{\faFileImage}}
\newcommand{\OA}{\thinspace\aiOpenAccess\enspace}

% Macros to set the year and duration on the left column
\newcommand{\Duration}[2]{\fontsize{9pt}{0}\selectfont #1 -- #2}
\newcommand{\Year}[1]{\fontsize{9pt}{0}\selectfont #1}
\newcommand{\Ongoing}{now}
\newcommand{\Future}{future}
\newcommand{\Appointment}[4]{\textbf{#1} \newline #2 \newline #3 \newline #4}

% Define command to insert month name and year as date
\usepackage{datetime}
\newdateformat{monthyear}{\monthname[\THEMONTH], \THEYEAR}

% Set the page margins
\usepackage[a4paper,margin=1.5cm,includehead,headsep=5mm]{geometry}

% To get the total page numbers (\pageref{LastPage})
\usepackage{lastpage}

% No indentation
\setlength\parindent{0cm}

% Increase the line spacing
\renewcommand{\baselinestretch}{1.2}
% and the spacing between rows in tables
\renewcommand{\arraystretch}{1.5}

% Remove space between items in itemize and enumerate
\usepackage{enumitem}
\setlist{nosep}

% Use custom colors
\usepackage[usenames,dvipsnames]{xcolor}

% Set fonts (requires compilation with xelatex)
\usepackage{fontspec}
\setmainfont[%
  Path = fonts/notoserif/,
  UprightFont = NotoSerif-Regular,
  BoldFont = NotoSerif-Bold,
  ItalicFont = NotoSerif-Italic,
  Extension = .ttf
]{NotoSerif}

% Set the spacing for sections
\usepackage{titlesec}
\titleformat{\section}
  {\normalfont\Large\mdseries} % format
  {} % label
  {0pt} % separation (left separation for hang)
  {} % text before title
  [\titlerule] % text after title
\titleformat{\subsection}
  {\normalfont\large\mdseries} % format
  {} % label
  {0pt} % separation (left separation for hang)
  {} % text before title

% Disable number of sections. Use this instead of "section*" so that the sections still
% appear as PDF bookmarks. Otherwise, would have to add the table of contents entries
% manually.
\makeatletter
\renewcommand{\@seccntformat}[1]{}
\makeatother

% Set fancy headers
\usepackage{fancyhdr}
\pagestyle{fancy}
\fancyhf{}
\lhead{\fontsize{9pt}{10pt}\selectfont
  \monthyear\today
}
\chead{
  \fontsize{9pt}{10pt}\selectfont
  \MyName
  \hspace{0.2cm} -- \hspace{0.2cm}
  \Title
}
\rhead{\fontsize{9pt}{10pt}\selectfont \thepage{} of \pageref*{LastPage}}
\renewcommand{\headrulewidth}{0pt}

% Metadata for the PDF output and control of hyperlinks
\usepackage[colorlinks=true]{hyperref}
\hypersetup{
  pdftitle={\MyName\ - \Title},
  pdfauthor={\MyName},
  linkcolor=blue,
  citecolor=blue,
  filecolor=black,
  urlcolor=MidnightBlue
}

%%%%%%%%%%%%%%%%%%%%%%%%%%%%%%%%%%%%%%%%%%%%%%%%%%%%%%%%%%%%%%%%%%%%%%%%%%%%%%%

\begin{document}

% No header for the first page
\thispagestyle{empty}

%%%%%%%%%%%%%%%%%%%%%%%%%%%%%%%%%%%%%%%%%%%%%%%%%%%%%%%%%%%%%%%%%%%%%%%%%%%%%%%

\begin{minipage}[t]{0.7\textwidth}
{\fontsize{22pt}{0}\selectfont\MyName}
\end{minipage}
\begin{minipage}[t]{0.3\textwidth}
  \begin{flushright}
    Last updated: \monthyear\today
  \end{flushright}
\end{minipage}
\\[-0.1cm]
\rule{\textwidth}{2pt}
\\[0.1cm]
\begin{minipage}[t]{0.7\textwidth}
    ORCID: \href{https://orcid.org/\ORCID}{\ORCID}
    \\
    Email: \href{mailto:\Email}{\Email}
    \\
    Website: \Website{\PersonalWebsite}
\end{minipage}
\begin{minipage}[t]{0.3\textwidth}
  \begin{flushright}
    \Affiliation
  \end{flushright}
\end{minipage}

%%%%%%%%%%%%%%%%%%%%%%%%%%%%%%%%%%%%%%%%%%%%%%%%%%%%%%%%%%%%%%%%%%%%%%%%%%%%%%%

\section{Present Research}

Microbial communities are the biological foundation of all ecosystems on Earth, and the structure of these communities are shaped largely by the ever-present viruses that infect and kill bacteria. Currently, I am researching the molecular 'defence systems' that have evolved in bacteria to protect against viral infections, with an emphasis on developing tools to identify and discover novel types of defence systems. Through understanding defence systems, we hope to more effectively utilise viruses to eliminate pathogenic bacteria in healthcare and agricultural applications.

%%%%%%%%%%%%%%%%%%%%%%%%%%%%%%%%%%%%%%%%%%%%%%%%%%%%%%%%%%%%%%%%%%%%%%%%%%%%%%%
\needspace{12em}
\section{Education}

\begin{EntriesTable}
  \Duration{2020}{2023}  &
  \textbf{PhD in Microbiology}, \UOO, New Zealand
  \newline
  \textit{Thesis: The antiviral defence systems of bacteria and archaea}
  \\
  \Duration{2019}{2020}  &
  \textbf{BSc (Hons; 1\textsuperscript{st} class) in Microbiology}, \UOO,
    New Zealand
  \newline
  \textit{Dissertation: Mesorhizobium symbiosis islands encode diverse bacteriophage defence systems}
  \\
  \Duration{2008}{2009}  &
  \textbf{BBiomedSc in Infection and Immunity}, \UOO, New Zealand
\end{EntriesTable}

%%%%%%%%%%%%%%%%%%%%%%%%%%%%%%%%%%%%%%%%%%%%%%%%%%%%%%%%%%%%%%%%%%%%%%%%%%%%%%%
\needspace{12em}
\section{Awards \& Honors}

\begin{EntriesTable}
  \Year{2022}  &
  New Zealand Microbiological Society \textbf{Student Travel Grant}
  (\$800) to present at the New Zealand Microbiological Society Conference,
  New Zealand
  \\
  \Year{2022}  &
  Department of Microbiology and Immunology \textbf{Student Travel Grant}
  (\$2,000) to present at the Viruses of Microbes Conference, Portugal
  \\
  \Year{2022}  &
  Division of Health Sciences \textbf{Student Travel Grant} (\$2,000) to
  present at the Viruses of Microbes Conference, Portugal
  \\
  \Year{2021}  &
  DT Jones Microbiology \textbf{Student Travel Grant} (\$470) to
  present at the Federation of Asian and Oceanian Biochemists and Molecular
  Biologists Congress, New Zealand (virtual)
  \\
  \Duration{2020}{\Ongoing}  &
  \UOO\ \textbf{PhD Research Scholarship} (\$76,500)
  \\
  \Year{2019}  &
  GlycoSyn \textbf{Summer Research Scholarship} (\$5,000)
  \newline
  \textit{Project: How do soil bacteria protect themselves against viruses?}
\end{EntriesTable}

%%%%%%%%%%%%%%%%%%%%%%%%%%%%%%%%%%%%%%%%%%%%%%%%%%%%%%%%%%%%%%%%%%%%%%%%%%%%%%%
\needspace{12em}
\section{Teaching}

\subsection{Undergraduate}

\begin{EntriesTableExtra}
  \Duration{2020}{2022}  &
  MICR336: Microbial Ecology
  \newline
  Teaching R programming for microbial community analysis (senior demonstrator)
  \newline
  \textit{\UOO, New Zealand}
  & ~
  \\
  \Duration{2020}{2022}  &
  MICR335: Molecular Microbiology
  \newline
  Teaching fundamental molecular microbiology techniques (demonstrator)
  \newline
  \textit{\UOO, New Zealand}
\end{EntriesTableExtra}

%%%%%%%%%%%%%%%%%%%%%%%%%%%%%%%%%%%%%%%%%%%%%%%%%%%%%%%%%%%%%%%%%%%%%%%%%%%%%%%

\needspace{12em}
\section{Student supervision}

\subsection{Honours}

\begin{EntriesTable}
\Year{2022}  &
  Joel Haste, \UOO, New Zealand.
  \newline
  Supported supervision of student's postgraduate lab work.
  \newline
  \textit{Dissertation: The type II Thoeris system has two distinct defence mechanisms}
\end{EntriesTable}

\subsection{Undergraduate}

\begin{EntriesTable}
\Duration{2021}{\Ongoing}  &
  Jai Tarn, \UOO, New Zealand
  \newline
  Assisted in introducing student to postgraduate study, with a focus on bioinformatics.
\end{EntriesTable}

%%%%%%%%%%%%%%%%%%%%%%%%%%%%%%%%%%%%%%%%%%%%%%%%%%%%%%%%%%%%%%%%%%%%%%%%%%%%%%%

\needspace{12em}
\section{Publications}

\subsection{Peer-reviewed Papers}

\begin{EntriesTableExtra}
\Year{2021}  &
  \OA
  \Me, \TCT, \YW, \BJP, \CWR, \PCF, \FLN, \SAJ.
  Identification and classification of antiviral defence systems in bacteria and archaea with PADLOC reveals new system types.
  \emph{Nucleic Acids Res}.
  \DOI{10.1093/nar/gkab883}
  &
  \Print{10.1093/nar/gkab883}
  \\
\Year{2022}  &
  \OA
  \Me, \SM, \MRM, \CP, \NT, \PCF, \SAJ.
  PADLOC: a web server for the identification of antiviral defence systems in microbial genomes.
  \emph{Nucleic Acids Res}.
  \DOI{10.1093/nar/gkac400}
  &
  \Print{10.1093/nar/gkac400}
\end{EntriesTableExtra}

\subsection{Open-source Software}

\begin{EntriesTableExtra}
  \Duration{2019}{\Ongoing}  &
  \textbf{PADLOC} | \Website{www.padloc.otago.ac.nz}
  \newline
  A tool and web server for identifying defence systems in microbial genomes
  \newline
  Role: Creator, main developer
  &
  \GitHub{padlocbio/padloc}
  \\
\end{EntriesTableExtra}

%%%%%%%%%%%%%%%%%%%%%%%%%%%%%%%%%%%%%%%%%%%%%%%%%%%%%%%%%%%%%%%%%%%%%%%%%%%%%%%

\needspace{12em}
\section{Presentations}

\subsection{Talks}

\begin{EntriesTableExtra}
\Year{2022}  &
  \Me.
  The uncharacterised genes embedded in defence systems encode new types of defence.
  \emph{University of Liverpool},
  Liverpool, United Kingdom.
  &
  \hspace{1cm}
  \\
\Year{2022}  &
  \Me.
  Expanding on the ever-growing arsenal of antiviral defences in prokaryotes.
  \emph{\UOO, Microbiology \& Immunology Postgraduate Symposium},
  Dunedin, New Zealand
\end{EntriesTableExtra}

\subsection{Posters}

\begin{EntriesTableExtra}
\Year{2022}  &
  \Me, \THughes, \PCF, \SAJ.
  Identification of new antiviral defence mechanisms to advance our understanding of bacterial immune systems.
  \emph{New Zealand Microbiological Society Annual Conference},
  Wellington, New Zealand.
  \\
\Year{2022}  &
  \Me, \TCT, \YW, \SM, \MRM, \CP, \NT, \BJP, \THughes, \CWR, \PCF, \FLN, \SAJ.
  Identification of CRISPR-Cas and novel phage defence systems to expand our molecular toolkit.
  \emph{Queenstown Research Week CRISPR Technologies Satellite},
  Queenstown, New Zealand.
  \\
\Year{2022}  &
  \Me, \TCT, \YW, \SM, \MRM, \CP, \NT, \BJP, \THughes, \CWR, \PCF, \FLN, \SAJ.
  The Prokaryotic Antiviral Defence LOCator (PADLOC) for the identification and discovery of diverse novel defence systems.
  \emph{Viruses of Microbes Conference},
  Guimarães, Portugal.
  \\
\Year{2021}  &
  \Me, \TCT, \YW, \BJP, \CWR, \PCF, \FLN, \SAJ.
  Identification and classification of antiviral defence systems in bacteria and archaea with PADLOC reveals new subtypes.
  \emph{Federation of Asian and Oceanian Biochemists and Molecular Biologists Congress},
  Christchurch, New Zealand.
  &
  \PosterDOI{10.6084/m9.figshare.17058113}
  \\
\Year{2020}  &
  \Me, \BJP, \CWR, \PCF, \SAJ.
  Mesorhizobium symbiosis islands encode diverse bacteriophage defence systems.
  \emph{Genetics Otago Symposium},
  Dunedin, New Zealand.
  &
  \PosterDOI{10.6084/m9.figshare.16442001}
  \\
\Year{2019}  &
  \Me, \BJP, \CWR, \PCF, \SAJ.
  Mesorhizobium symbiosis islands encode diverse bacteriophage defence systems.
  \emph{Microbiology and Immunology, and Biochemistry Research Symposium},
  Dunedin, New Zealand.
  &
  \PosterDOI{10.6084/m9.figshare.16442001}
\end{EntriesTableExtra}

%%%%%%%%%%%%%%%%%%%%%%%%%%%%%%%%%%%%%%%%%%%%%%%%%%%%%%%%%%%%%%%%%%%%%%%%%%%%%%%
\section{Miscellaneous}

\subsection{Professional society affiliations}

\begin{EntriesTable}
  \Duration{2022}{\Ongoing} & Maurice Wilkins Centre (MWC) affiliate investigator
  \\
  \Duration{2021}{\Ongoing} & American Society for Microbiology (ASM) member
  \\
  \Duration{2020}{\Ongoing} & New Zealand Microbiological Society (NZMS) member
  \\
  \Duration{2020}{\Ongoing} & New Zealand Society for Biochemistry and Molecular Biology (NZSBMB) member
\end{EntriesTable}

%%%%%%%%%%%%%%%%%%%%%%%%%%%%%%%%%%%%%%%%%%%%%%%%%%%%%%%%%%%%%%%%%%%%%%%%%%%%%%%
\section{Glossary}

These are the meanings of the symbols used throughout this document:
\\
\TablePad
\begin{tabularx}{\textwidth}{@{}p{0.03\textwidth} p{0.97\textwidth}@{}}
  \aiOpenAccess & Indicates that a publication is open-access
  \\
  \faGithub & Link to a code repository on GitHub
  \\
  \faFilePdf & Link to an open-access PDF
  \\
  \faImage & Link to a poster
\end{tabularx}

\end{document}
